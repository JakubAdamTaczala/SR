\subsection{Idea robota}
Wzorem do stworzenia robota są pojazdy mobilne służące do inspekcji rur. Jest to bardzo ważny aspekt ze względu na to, iż takie pojazdy są wstanie skontrolować kanalizacje miejską bez konieczności niszczenia otoczenia w skutek wykopów, oraz rozbierania rur.
\subsection {Zadania robota}
\begin{itemize}
\item Poruszanie się po poziomym korytarzu o szerokości 30cm posiadającym gładką strukturę;
\item Poruszanie się autonomiczne gdy trasa będzie jednoznaczna i nie będzie na niej zakrętów o kącie większym niż 90 stopni;
\item Wysyłanie danych z akcelerometru i żyroskopu;
\item Wykrywanie stężenia metanu.
\end{itemize}

\subsection {Wyposażenie robota}
\begin{itemize}
\item Czujniki odległości;
\item Akcelerometr;
\item Żyroskop;
\item Silniki;
\item Moduł radiowy;
\item Serwomechanizmy;
\item Czujnik metanu.
\end{itemize}

\subsection{Sterowanie alternatywne}

Robot będzie również sterowany poprzez Joysticki, jest to ważna funkcja, by operator mógł przejąć kontrolę ze względu na możliwość wystąpienia niejednoznacznej trasy. Jednocześnie obserwując ekran komputera operator będzie miał wgląd w podstawowe informacje takie jak przyśpieszenie, nachylenie oraz stężenie metanu.